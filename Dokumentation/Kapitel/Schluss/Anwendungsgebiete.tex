\section{Weiterführende Anwendungsgebiete}
Es stellt sich nun die Frage, welchen technischen Nutzen die Teleoperation innerhalb der Gesellschaft finden könnte.
\\
\\
\noindent
Zunächst gibt es ein großes Anwendungsgebiet in der Medizin. Die Idee Roboter in der Medizin einzusetzen ist nicht neu. Gerade um die Arbeit bei Operationen zu erleichtern bzw. sicherer im Hinblick auf Komplikationen zu machen, werden neue Einsatzmöglichkeiten von Robotern evaluiert. Eine Teleoperation könnte dazu dienen, die Operation quasi nur virtuell durchzuführen und den physikalischen Eingriff dem Roboter zu überlassen. Dies hätte den Vorteil, dass menschliche Einflussfaktoren herausgefiltert werden. Das könnte Beispielsweise das Zittern eines Operateurs oder der "`schlechte"' Tag eines jenen sein.

Doch genau so gibt es die Einsatzmöglichkeiten im Rahmen der Delaborierung (Entschärfung) von Bomben bzw. scharfen Einheiten in Einzelteile, so dass diese nicht mehr gefährlich sind. Die Idee ist dabei, die gefährliche Nähe zur Bombe oder scharfen Einheit an den Roboter zu delegieren und die Delaboration aus sicherer Entfernung durchzuführen. Dabei wirkt der Roboter wie oben im Beispiel der Medizin als Filter der menschlichen Einflussfaktoren. Ein menschlicher Delaborierer steht unter enormer Anspannung und Druck, ein Roboter hingegen kann die Aufgabe ohne psychologische Einflüsse durchführen. 

Diesen beiden doch positiven Beispielen steht jedoch auch ein Einsatz im militärischen Bereich gegenüber. Schon jetzt werden Teleoperation im Bereich von Kampf - Drohen von fast allen Militärs verwendet. Da liegt es auf der Hand, dass ein geeignetes System menschliche Bewegungen auf einen Roboter zu übertragen auch im militärischen Bereich eingesetzt werden könnte. Unserer Meinung nach ist eine Entwicklung in diese Richtung sehr kritisch zu beäugen, da dadurch das Krieg führen nur noch vor dem Bildschirm stattfinden würde.
\\
\\
\noindent
So ist zusammenfassend zu Betrachten, dass ein Einsatz von Teleoperationen in vielen Bereichen die heutige Arbeit leichter und vor allem sicherer machen würde. Jedoch ist nicht auszulassen, dass eine solche Entwicklung auch negative Aspekte mit sich bringt.

%Medizin, Bombenentschärfung, leider auch Militär 