\chapter{Schlussbemerkung}
Dieser Abschnitt dient als knappe Zusammenfassung und Fazit über die Arbeit "`\Titel"'. Dabei wird aufgezählt, welche weiteren Funktionen hinzugefügt werden könnten und in welchem Anwendungsgebiet solche Gestiksteuerungen durch Roboter denkbar wären.

Nachdem in den Kapiteln \myref{c:robotik}, \myref{c:kinect} und \myref{c:nao} die Grundlagen für das Verständnis dieser Arbeit gelegt wurden, ist in Abschnitt \myref{c:realisierung} die konkrete Implementierung beschrieben. In den Grundlagenkapiteln wurde unter anderem der Begriff mobile Roboter definiert, ein kurzer Rückblick auf die Geschichte dieser gegeben und eingeteilt aus welchen Komponenten diese bestehen. In den Kapiteln zum Roboter Nao und der Kinect Kamera wurde bei beiden ein Überblick über die Hard- und Software gegeben und die Programmierschnittstellen vorgestellt.

Die Architektur inklusive der Schnittstellen für leichtes Austauschen von konkreten Implementierungen ist in Kapitel \ref{k:Architektur} vorgestellt. Dabei wurden zentrale Algorithmen und Prototypen der Anwendung genannt und erläutert. Zudem wurden Erweiterungen vorgestellt, welche die Anwendung verbesserten.


\section{Ausblick}      % Kapitelname
Das Ergebnis dieser Arbeit ist eine Anwendung, die Bewegungen des Menschen erkennt, annimmt, diese passend für den Roboter umrechnet und sie dem Roboter zur Ausführung übergibt. Der primäre Meilenstein \textit{Steuerung der Arme} wurde zu 100\% erreicht. Als weitere Meilensteine würden sich unserer Meinung nach folgende anbieten: 
\begin{itemize}
\item Steuerung der Beine: Durch einen Bewegungskreis könnte die Kinect erkennen, ob sich der Mensch  nach rechts/links oder vorne/hinten bewegt hat und dies entsprechend dem Roboter weitergeben
\item Sonderbefehle mit Gesten: Bestimmte Gesten ermöglichen erweiterte Funktionalitäten anstatt simpler Imitation. Beispielsweise das Zusammenführen der Hände löst eine Kameraaufnahme aus. 
\item Steuerung durch Sprache: Die integrierte Sprachbibliothek der Kinect könnte für weitere Funktionen genutzt werden, wie zum Beispiel das Verändern der Pose.
\end{itemize}
Die Performance der Anwendung könnte durch Einsetzen von Rotationsmatrizen in den Algorithmen eventuell gesteigert werden.
\\
\\
\noindent
Ebenso könnte die Genauigkeit der Imitationen durch Einsatz der neuen Kinect (Xbox ONE) deutlich gesteigert werden. Diese identifiziert das Skelett noch genauer und ist auch in der Berechnung der Winkel hochauflösender. Zu weiteren Funktionalitäten gehört, dass sogar eine Rotation der Unterarme, sowie das Öffnen bzw. Schließen der Hände erkannt wird. Dies würde die Implementierung zwar deutlich komplexer machen, allerdings hätte die fertige Anwendung danach einen höheren Mehrwert.

%
%
% Sammlung aller aufgetretener Probleme während der Implementierung
%
%
%	Kinect Beendet nicht richtig
%		-> Eigene Klasse erstellt, die herunterfaehrt = Known Issue Microsoft
%
%
%
%	Ruckeln im Nao Arm
%		-> Kinect Filter (gleitender Mittelwertfilter über letzen 3 Werte)
%		-> Nao Speed geregelt (Geschw. zu den Winkeln)
%

\section{Aufgetretene Probleme}
\section{Weiterführende Anwendungsgebiete}
Es stellt sich nun die Frage, welchen technischen Nutzen die Teleoperation innerhalb der Gesellschaft finden könnte.
Zunächst gibt es ein großes Anwendungsgebiet in der Medizin....

%Medizin, Bombenentschärfung, leider auch Militär 

	