\section{Überblick}
In den folgenden Abschnitten wird das Thema Scheduling näher betrachtet. \\
Im Abschnitt \myref{ch:scheduling} wird dieser Ausdruck allgemein definiert, zentrale Kriterien vorgestellt und auf unterschiedliche Verfahren und Problem hingewiesen.\\
Die Passage \myref{ch:schedInformatik} erklärt beispielhaft drei Anwendungsgebiete von Scheduling. Dies beinhaltet auch die Definition einiger Strategien (Algorithmen), die für optimales Scheduling entwickelt wurden. Diese werden Anhand des Betriebssystem-Scheduling erläutert. \\
Ferner wird im Kapitel \myref{ch:enterprise} das \textit{normale} mit dem \textit{Enterprise}-Scheduling verglichen und die Gemeinsamkeiten und Unterschiede deutlich gemacht.  Dabei wird verdeutlicht, warum moderne IT-Firmen Enterprise-Scheduling brauchen und wie es am Beispiel der \ac{Fiducia} mit \textit{Cronacle v9\footnote{Scheduling-Software der Firma \textit{Redwood} für Open-Systems}} angewendet wird. 
TODO:SCHLUSS
