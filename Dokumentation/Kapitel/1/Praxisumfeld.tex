\section{Praxisumfeld}
Die Fiducia - Gruppe ist einer der größten IT - Dienstleister in Deutschland und der größte in der genossenschaftlichen FinanzGruppe. 
Mit ihren Tochter- und Beteiligungsunternehmen bietet die \ac{Fiducia} auf dem Gebiet der Informationstechnologie ein umfassendes Dienstleistungsspektrum an.\\
Das Kerngeschäft ist hierbei die Erbringung von IT - Dienstleistungen für rund 800 Banken, Instituten und Unternehmen in der genossenschaftlichen FinanzGruppe.
Der auf höchstem Sicherheitsniveau stattfindende Rechenzentrumsbetrieb unter Einsatz von neuster Großrechner-, Unix- und Windows- Technologie und die Entwicklung
sowie die Implementierung zugehöriger IT - Lösungen gehören zu den Kernkompetenzen der Fiducia. \cite{testitest} \\
Der Bereich \ac{ITS} ist dafür verantwortlich, die gesamte IT - Infrastruktur und die darauf basierenden IT - Anwendungen für das Bankensystem \textit{agree}, 
für den Privatbanken - Bereich sowie für Wirtschaft und Verwaltung bereitzustellen. Der Betrieb findet dabei in zwei redundanten Rechenzentren am Hauptsitz Karlsruhe statt. Allein im Bereich der Volks- und Raiffeisenbanken nutzen ca. 100.000 Arbeitsplätze Anwendungen, die das Fiducia Rechenzentrum bereitstellt. Über alle Kunden und Anwendungen hinweg, nutzt die Fiducia über 8.000 Unix basierte Server, um das reibungslose Tagesgeschäft sicherzustellen.\\
Die Abteilung \ac{IS25TO} ist dabei für die Bereitstellung und den Betrieb von Transaktionsmonitoren\footnote{Die Hauptaufgabe von Transaktionsmonitoren ist das Abwickeln von Interaktionen zwischen Anwendungen auf heterogenen Systemen} und Operation - Tools zuständig. Zu den Transaktionsmonitoren gehören IMS DB/DC, CICS und Websphere MQ. Zu den Operation - Tools zählen die Scheduler - Systeme zentral auf dem Mainframe und dezentral in der Unix und Windows - Welt und eine Filetransfer - Infrastruktur. Neben der Bereitstellung und des Betriebs gehört zu den Aufgaben der Abteilung auch die Beratung und Schulung sowohl von intern als auch extern.\\
Das Team Scheduling betreibt und supportet die beiden Tools \textit{Tivoli Workload Scheduler} (Mainframe) und \textit{Cronacle} (Unix und Windows). 
Diese dienen zur plattformübergreifenden Steuerung und Überwachung von Batch - Anwendungen.\\
Darüber hinaus sind die Mitarbeiterinnen und Mitarbeiter der \ac{IS25TO} auch noch im 1st und 2nd Level Support eingesetzt. Dies findet während den normalen Arbeitszeiten, als auch im Bereitschaftsdienst außerhalb der regulären Arbeitszeit statt.
