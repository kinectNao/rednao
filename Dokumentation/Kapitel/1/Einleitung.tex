\chapter{Einleitung}      % Kapitelname
In dieser Einleitung wird das zentrale Thema dieser Arbeit beleuchtet, beschrieben in welchem Umfeld die Arbeit entworfen wurde und wie sich dieses Dokument inhaltlich aufbaut.

\section{Umfeld}
Diese Studienarbeit mit dem Thema \textit{\titel} wurde während zwei Theoriephasen an der Dualen Hochschule Baden-Württemberg am Standort Karlsruhe von Michael Stahlberger und Lukas Essig durchgeführt. Betreut wurde die Arbeit durch Herrn Prof. Hans-Jörg Haubner, sowie Herrn Michael Schneider. 

\section{Thema}
Zentrales Thema der Arbeit ist die Interaktion zwischen Mensch und Roboter. Dabei soll ein Modell des humanoiden Roboters Nao durch eine Gestik seines Benutzers gesteuert werden. Dabei ahmt der Roboter die Bewegungen nach, die der Benutzer zeitgleich bzw. etwas zuvor ausgeführt hat. Zudem sollen bestimmte Steuergesten implementiert werden, die den Roboter in verschiedene Modi versetzen, in denen er gleiche Gesten auf verschiedene Wege interpretiert. Die Gestik-Erkennung soll dabei mit Hilfe eines Xbox - Kinect Moduls erfolgen. Von dieser Stereokamera werden bestimmte Gesten erkannt und über eine Software verarbeitet und an den Nao-Roboter weitergeleitet, der diese dann nachahmt. Dabei ist natürlich darauf zu achten, dass keine Bewegungen von Nao ausgeführt werden, die ihm mechanisch oder elektronisch Schaden zufügen können.

Das Vorgehen ist folgendermaßen strukturiert: Nach der Einarbeitung in die Materie Nao - Roboter und Kinect - Kamera soll die Wahl einer geeigneten Programmiersprache getroffen werden. Daraufhin müssen zwei verschiedene Anwendungen konzeptioniert und implementiert werden. Erstens die Erkennung der Gesten mittels Kinect und zweitens das Empfangen der Gesten und die kinematische Umsetzung durch den Roboter. Zwischen diesen beiden Anwendungen muss eine geeignete Kommunikationsschicht entworfen und implementiert werden.

\section{Inhalt}


\todo{
Inhalt der Studienarbeit
}
