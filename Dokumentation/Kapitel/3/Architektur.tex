\section{Architektur}
Die Architektur der Anwendung kann grob in zwei Teilbereiche gegliedert werden, den Sensorteil und den Aktorteil.
Der Sensorteil ist dafür zuständig, die entsprechenden Gesten des Benutzers zu ermitteln und zu verarbeiten. Der Aktorteil der Anwendung ist dafür zuständig die Werte entsprechend auf die Wertebereiche des Nao-Koordinatensystems zu transformieren und diese dem Roboter zu übermitteln.

\todo{Architekturidee: Input, Verarbeitung(evtl. Filter->Probleme Ruckeln?), Output, }
\todo{Klassendiagramm}
\todo{Sequenzdiagramm eines bestimmten Anwendungsfalls}

