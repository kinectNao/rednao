\section{Kinect SDK}
Nach dem Erscheinen der Hardware war diese auch schnell in Entwicklerkreisen gefragt. Doch Microsoft selbst gefiel dies zunächst nicht, denn der Konzern befürchtete, dass Cheater sich an ihren Spielen zu schaffen machen würden.
So gab es zunächt kein SDK von Microsoft selbst. Die Open Source Gemeinde jedoch erkannte das Potential des Produktes schneller und entwickelte (eine Schnittstelle? zu) OpenNI, einem Framework, das die Auswertung von 3D-Sensordaten verschiedenster Hersteller unterstützt. Dieses Framework bietet somit durch seine Plattformunabhängigkeit die Möglichkeit unterschiedliche Betriebssysteme mit unterschiedlichen 3D-Sensoren zu kombinieren.\cite{webb2012beginning}
Microsoft zog nach und gab am 17. Juni 2011 die freie Beta Version des Microsoft SDKs frei. Somit hatte nun jeder Entwickler freien Zugang zu allen Kinectfunktionen, die auch von Microsoft selbst bisher genutzt wurden. Einer der Vorteile des Kinect SDKs besteht darin, dass die Skelett-Erkennung (siehe Kapitel \ref{Software} \nameref{Software}) ohne initiale Pose möglich ist, was im Gegensatz zum OpenNI-Framework steht. \cite{webb2012beginning} Da für dieses Projekt die Plattformunabhängigkeit nicht relevant ist, sowohl aber die Skeletterkennung, wurde sich für das Microsoft SDK entschieden.

\subsection{Richtiger Umgang mit dem SDK}



\todo{evtl. vergleich SDK OpenNI?}

