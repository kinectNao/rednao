\section{Freiheitsgrad}
Der Begriff Freiheitsgrad oder DOF (engl.: Degree(s) of Freedom) bezeichnet die Zahl von voneinander unabhängigen Bewegungsmöglichkeiten eines Systems. Ein starrer Körper im Raum hat beispielsweise einen Freiheitsgrad von sechs. Er kann sich in drei voneinander unabhängigen Richtungen und Ebenen bewegen beziehungsweise drehen. Als \textit{Freiheiten} werden die einzelnen Bewegungs- bzw. Rotationsmöglichkeiten bezeichnet. Somit hat der starre Körper je drei Translations- und Rotationsfreiheitsgrade. 

\cite{hertzberg2009mobile} definiert Freiheitsgrad in zwei Ausprägungen: 
\begin{itemize}
\item \textbf{Aktiver Freiheitsgrad} bezeichnet die Zahl der translatorischen wie rotatorischen Bewegungen, die eine Gelenk oder eine Roboterkomponente ausführen kann. Dies deckt sich mit der Definition von oben.
\item \textbf{Effektiver Freiheitsgrad} bezieht sich auf die \textit{Pose}. Sprich die Dimensionierung der Position und Orientierung, die ein Roboter einnehmen kann.
\end{itemize}
So hat beispielsweise ein differentialgetriebener Roboter, der sich nur in einer Ebene bewegen kann zwei aktive, dagegen aber drei effektive Freiheitsgrade.

Wenn ein Robotersystem direkt von jeder möglichen Pose in jede andere mögliche Pose wechseln kann, wird von einem \textit{holonomen} Robotersystem gesprochen. Solche Roboter sind beispielsweise diejenigen, die in der Lage sind in jede beliebige Richtung zu fahren, ohne davor eine Drehung machen zu müssen. Als allgemeine Regel gilt: \\
$ nicht-holonom = DOF_{effektiv} > DOF_{aktiv} $ \\
Das heißt, ein System ist demzufolge auf jeden Fall nicht holonom, falls die Anzahl der effektiven Freiheitsgrade größer wie die der aktiven ist.