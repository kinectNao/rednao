%%
%
%	Architektur		
%
%%
\section{Architektur}\label{k:Architektur}
Die Architektur der Anwendung kann grob in zwei Teilbereiche gegliedert werden, den Sensorik- und den Aktorik-Bereich.
Der Sensorteil ist dafür zuständig, die entsprechenden Gesten des Benutzers zu ermitteln und zu verarbeiten. Bei der Digitalisierung der Gesten werden diese mit einem Medianfilter vorgefiltert, sodass extreme Ausreiser verschwinden, die ungewollte Bewegungen des Roboters zur Folge hätten.
Der Aktorteil der Anwendung transformiert die Posen der Armgelenke in den Nao-Wertebereich und sendet diese an den Roboter.

%
% Programm statisch erklaeren, einzelne Module
%
% Erklärung der einzelnen Module + was ist der Knackpunkt?
%		MainWindow -> starten & initialisieren
%			!Annotation! := Erläutern Thread Konzept C# .xaml + .xaml.cs
%							bzw. Dispatcher (security != Java)
%		Interfaces 		-> 	austauschbar & fokus auf relevante Daten => Winkel
%							Wiederverwendbar für Nao UND gui
%		KinectHandler	-> 	Input der Anwendung
%		Calculation 	-> 	Threading, Berechnungen CPU-Intensiv
%						-> 	Errechnet aus Skelett Winkel
%						-> 	Versorgt GUI, Nao mit Winkelwerten
%						-> 	Werte vorfiltern (Filter erklären)
%		NaoHandler 		-> 	Output der Anwendung
%							Mapping Kinect-Nao-Raum, Ruckeln vermeiden
%

Klassendiagramm\todo{Klassendiagramm einfuegen}

\subsection{MainWindow}
Die C\# Applikation wird als WPF-Anwendung gestartet. Dies geschieht in der Klasse \textbf{MainWindow}. Hierbei werden alle Parameter initialisiert und alle nötigen Objekte erzeugt, wie der KinectSensor, der KinectHandler und der NaoHandler. Es wird wie in Kapitel ...SDK... beschrieben, der Tiefenstream und der Skeletonstream der Kinect aktiviert und anschließend ein "AllFramesReady"-Event registriert, das benutzt wird, sobald ein Skelett vor der Kamera volständig erkannt wurde. 


\subsection{Interfaces}
Als Schnittstelle zwischen der Erkennung und der Ausführung der Armpositionen wird essenziell die Methode \textit{updateAngles} vom Interface \textit{ISkeletonAngles} verwendet. Diese stellt alle benötigten Winkel wie Shoulder Pitch, Shoulder Roll, Ellbow Roll, Elbow Yaw bereit. Somit kann dieses Interface von allen Klassen implementiert werden, die immer die aktuellen Werte benötigen, wie der NaoHandler und die GUI.


%
% Dynamische Erklaerung des Programmes -> Workflow
%
%	Workflow darstellen (evtl. Flussdiagram + Erläuterung)
%		Programm starten über MainWindow
%		Erzeugt Frame und startet/initialisiere alle Submodule
%			Init KinectHandler
%			Init NaoHandler
%			Init Berechnungen
%		Warte auf Skelett...
%		Skelett erkannt -> starte Berechnungen
%		Zeige Skelett im Hauptframe an
%		Für jedes aktualisierte Skelett
%			Errechne Winkel
%			Winkel filtern
%			Aktuellen Status aktualisieren -> Observer Pattern
%				GUI
%				NaoHandler
\subsection{Flussdiagramm: Winkelerkennung}



\todo{Architekturidee: Input, Verarbeitung(evtl. Filter->Probleme Ruckeln?), Output, }