\section{Architektur}
Die Architektur der Anwendung kann grob in zwei Teilbereiche gegliedert werden, den Sensorik- und den Aktorik-Bereich.
Der Sensorteil ist dafür zuständig, die entsprechenden Gesten des Benutzers zu ermitteln und zu verarbeiten. Der Aktorteil der Anwendung ist dafür zuständig die Werte entsprechend auf die Wertebereiche des Nao-Koordinatensystems zu transformieren und diese dem Roboter zu übermitteln.


\subsection{Interfaces}
Als Schnittstelle zwischen der Erkennung und der Ausführung der Armpositionen wird essenziell die Methode \textit{updateAngles} vom Interface \textit{ISkeletonAngles} verwendet. Diese stellt alle benötigten Winkel wie Shoulder Pitch, Shoulder Roll, Ellbow Roll, Elbow Yaw bereit. Somit kann dieses Interface von allen Klassen implementiert werden, die immer die aktuellen Werte benötigen, wie der NaoHandler und die GUI.

\subsection{Klassendiagramm}



\subsection{Sequenzdiagramm: Winkelerkennung}



\todo{Architekturidee: Input, Verarbeitung(evtl. Filter->Probleme Ruckeln?), Output, }