\section{Nao-Armbewegungsalgorithmus}\label{s:naoAlgo}
Der zweite wesentliche Teil des Programms beinhaltet die Übertragung der erhaltenen Winkel der Kinect auf den Roboter. Dies ist nicht einfach nur die Weiterleitung der Werte sondern bietet auch verschiedene Funktionen, wie \textit{Umrechnen} oder \textit{Validieren}. Die Basisalgorithmen hierzu werden im Folgenden vorgestellt.
\\
\\
\textbf{Basis}
\\
Das Interface \textsf{Arm} deklariert eine Methode \textsf{controlArm} mit den Parametern für die fünf Armwinkel. Die erbenden Klassen \textsf{LArm} und \textsf{RArm} implementieren diese Methode. 

\lstinputlisting
    [caption={Methode \textsf{controlArm}},captionpos=b,
       label=lst:nao_controlArm,
       ]	
 {Listings/controlArm.cs}

Das Listing \ref{lst:nao_controlArm} zeigt die Definition der Funktion \textsf{controlArm}. Wie im Methodenkopf zu sehen, werden hier fünf Winkel übertragen. Auch wenn WristYaw (WY) nicht durch die Kinect erkannt wird, ist dieser Winkel zur Vollständigkeit trotzdem angegeben. 

In Zeile vier werden die Winkelwerte an die Funktion \textsf{convertAngles} übergeben. Diese sorgt dafür, dass die Kinect-Winkel in die passenden Nao-Winkel umgerechnet werden. So ist beispielsweise ein Winkel von $90^\circ$ des ShoulderPitch-Gelenk im Nao-Gelenkraum $0^\circ$. Die Funktion gibt am Ende ein Array der umgerechneten Werte zurück.

Dieses wird in Zeile fünf an die Funktion \textsf{verifyAngles} übergeben. Diese überprüft für jeden Winkel, ob dieser im Nao-Gelenkraum liegt (Siehe Kapitel \ref{nao:hardware}-Gelenkraum). Liegt ein Wert nicht im gültigen Bereich, wird der ungültige Wert mit dem Wert der aktuellen Winkelstellung überschrieben.

Nach den Funktionen für die Umrechnung und die Validierung der Winkel wird in Zeile sieben der Befehl zum Bewegen des Arms an Nao geschickt. Dafür ist die Funktion \textsf{setAngles} verantwortlich. Neben den Winkeln (\textsf{newangles}) und den Namen der betroffenen Gelenke (\textsf{joints}) wird zusätzlich noch der Wert \textsf{Arm.fractionMaxSpeed} übergeben. Dieser gibt an, mit welcher Geschwindigkeit in Prozent die Gelenke an die Winkelstellungen gefahren werden sollen. Nach subjektiven Tests wurde hier der Wert 0.3f (30\%) gewählt. 
\\
\\
\textbf{Erweitert}
\\
Da das ständige Neuausrichten der Winkel für die Motoren der Gelenke sehr anfordernd ist, wurden neben dem Glätten der Kinect-Winkel durch den Median-Filter auch noch im Armbewegungsalgorithmus zwei Erweiterungen implementiert.

\lstinputlisting
    [caption={Methode \textsf{controlArm erweitert}},captionpos=b,
       label=lst:nao_checkDifference,
       ]	
 {Listings/controlArm2.cs}

In Listing \ref{lst:nao_controlArm2} ist zu sehen, dass der Algorithmus um eine Funktion erweitert und eine andere geändert wurde.

Der Pseudo-Code der Methode \textsf{checkDifference} (Zeile 6) ist in Listing \ref{lst:nao_controlArm2} zu sehen. Darin wird überprüft, ob die Differenz zwischen dem neuen Winkel und dem aktuellen Winkel kleiner als eine Schranke ist. Ist jenes nicht der Fall, wird der neue Winkel übernommen, ansonsten bleibt der aktuelle Winkel stehen. 
Als Schranke wurde für ElbowRoll $5^\circ$ und die anderen Winkel $10^\circ$ gewählt. ElbowRoll hat eine niedrigere Schranke, da der Gelenkraum dieses Gelenks deutlich kleiner ist, wie die der restlichen Winkel (Siehe \ref{nao:hardware}-Gelenkraum). 
\lstinputlisting
    [caption={Pseudocode \textsf{checkDifference}},captionpos=b,
       label=lst:nao_controlArm2,
       ]	
 {Listings/checkDifference.cs}
Diese Überprüfung soll verhindern, dass sich kleine Änderungen der Kinect-Winkel, also des Menschen, direkt auf die Armstellungen Naos übertragen. Die Werte der Schranken wurden so gewählt, dass die Armstellung Naos subjektiv zu der des Menschen passt.


In Listing \ref{lst:nao_controlArm2} wurde die Methode \textsf{setAngles} durch \textsf{post.angleInterpolation
WithSpeed} ersetzt. Prinzipiell machen beide Methoden das Gleiche und zwar bewegen sie die übergebenen Gelenke an die übergebenen Winkel mit der übergebenen Geschwindigkeit. Der große Unterschied liegt allerdings darin, dass \textsf{setAngles} eine Non-Blocking- und \textsf{post.angleInterpolationWithSpeed} eine Blocking-Funktion ist. In diesem Fall ist eine Non-Blocking-Funktion schlecht, da sie es erlaubt, dass die ausführende Methode zu jedem Zeitpunkt unterbrochen  und mit neuen Parametern erneut gestartet werden kann. Unter anderem führt das dazu, dass die Arme "`ruckeln"' wenn sie bewegt werden. Die Blocking-Funktion wird als Folge erst erneut aufgerufen, wenn die zuvor ausgeführte Methode beendet wurde. Dies entspricht auch eher dem Ziel, dass die Bewegungen des Menschen \textbf{nachgeahmt} werden und es keine eins zu eins Spiegelung des Menschen auf den Roboter ist.

 
